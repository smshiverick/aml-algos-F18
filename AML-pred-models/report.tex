 \\\documentclass[sigconf]{acmart}

\input{format/final}

\begin{document}
  \title{Comparing Predictive Models of Pain Reliever Misuse and Abuse}
  \author{Sean M. Shiverick}
  \affiliation{
  \institution{Indiana University-Bloomington}
  }
\renewcommand{\shortauthors}{S.M. Shiverick}

%%%%%%%%%%%%%%%%%%%%%%%%%%%%%%%%%%%%%%%%%%%%%%%%%%%%%%%%%%%%%%%%%%%%%%%%%%%%%%%%

\begin{abstract}

The misuse and abuse of prescription opioids (MUPO) is a chronic health 
condition associated with increased overdose deaths in recent years 
\cite{nida18}. Predictive modeling can identify factors related to MUPO and 
help predict individuals at risk for opioid addiction. The present study 
compares ten linear classification models using four performance metrics 
: accuracy, sensitivity (recall), precision, and $f_1$-score. The sample 
data consisted of N = 114,038 respondents from the National Survey on Drug 
Use and Health (NSDUH) for 2015 and 2016. Of the total sample, 27\% had 
used any opioid pain reliever medication in the past year, and 11\% of 
respondents had previously misuse or abused pain relievers; 2\% of the 
sample reported using heroin. The classifier models were fit to a training 
set with fifteen predictor variables and pain reliever misuse and abuse as 
the target outcome. Class labels were predicted for the testing set and model 
performance was evaluated in a confusion matrix. Neural networks (MLP) and 
support vector classifier had the highest model accuracy; however, logistic
regression, decision trees, and random forests, had the highest $f_1$-score. 
The $f_1$-score is considered a better metric of performance than accuracy 
With data that has unbalanced unbalanced classes. Advantages and limitation 
of the classification models are discussed. 
\footnote{ Address correspondence to \textit{smshiver@iu.edu}}

\end{abstract}
\keywords{Predictive Modeling, Supervised Learning, Classification Models}
\maketitle

%%%%%%%%%%%%%%%%%%%%%%%%%%%%%%%%%%%%%%%%%%%%%%%%%%%%%%%%%%%%%%%%%%%%%%%%%%%%%%%%
\section{Introduction}

Over the past two decades the misuse and abuse of prescription opioids (MUPO)  
has become a major health crisis in the U.S. \cite{volkow14}. In 2015, an 
estimated 2 million Americans suffered a substance use disorder related 
to prescription opioid pain relievers such as oxycodone or hydrocodone 
in 2015 \cite{nida18}. Opioid dependence and relapse are chronic health 
conditions and following treatment many addicted individuals are at high 
risk for relapse and overdose death \cite{shaham03}. The number of opioid-
related overdose deaths has more than quadrupled from 1999 to 2016. 
Each day, an average of 115 people die from an opioid overdose in the U.S. 
\cite{cdc18, judd16}. Supply-based interventions to reduce the availability 
of prescription opioids have produced a shift to the use of heroin and 
synthetic opioids such as fentanyl \cite{jones15}. The risk of overdose 
death from illicit and synthetic opioids is greatly increased because 
the dosage levels and potency are largely unknown. The sharp rise in 
prescription overdose deaths (POD) and heroin overdose deaths (HOD) are 
correlated \cite{muhuri13, unick13}. Predictive modeling approaches can 
help identify individuals susceptible for opioid addiction and may provide 
insights that inform policy decisions for addressing the opioid crisis. 
This study compares different classification models of pain reliever misuse 
and abuse and identifies several important features that contribute to 
the misuse and abuse of prescription opioids. 

%%%%%%%%%%%%%%%%%%%%%%%%%%%%%%%%%%%%%%%%%%%%%%%%%%%%%%%%%%%%%%%%%%%%%%%%%%%%%%%%

\subsection{Predictive Modeling}

Predictive modeling, statistical learning, or machine learning describe a 
set of procedures and automated processes for extracting knowledge from data 
\cite{james13, kuhn13, muller17, raschka17}. The two main branches of 
predictive modeling are supervised learning and unsupervised learning. 
Supervised learning problems involve prediction about a specific target 
variable or outcome. If a given dataset has no target outcome, unsupervised 
learning methods can be used to discover underlying structure in unlabeled data 
(e.g., clustering). Supervised learning is used to predict a certain outcome 
from a given input, when examples of input/output pairs are available in the 
data (e.g., logistic regression) \cite{muller17}. A statistical learning model 
is constructed on a set of observation used to train the model set and can then 
be used to predict new observations. Two major approaches to supervised learning 
problems are regression and classification. When the target variable to be 
predicted is continuous, or there is continuity between the outcome 
(e.g., home values), a regression model is used to test the set of features 
that predict the target variable. If the target is a class label, set of 
categorical or binary outcomes (e.g., spam or ham emails, benign or malignant 
cells), then classification is used to predict which class or category label 
that new instances will be assigned to. The present study uses a supervised 
learning approach to classify instances of pain reliever misuse and abuse 
using several classification models. 

%%%%%%%%%%%%%%%%%%%%%%%%%%%%%%%%%%%%%%%%%%%%%%%%%%%%%%%%%%%%%%%%%%%%%%%%%%%%%%%%

In the era of ``big data'', large amounts of health information are being 
generated from electronic medical records (EMRs), clinical research data, to 
population-level health data \cite{herland14}. Although it can be difficult 
to obtain reliable information about opioid use based on self-reports, surveys 
provide data on a range of issues that people may be reluctant to disclose 
such as illicit drug use and mental health problems. The data for the present 
study was obtained from the National Survey on Drug Use and Health (NSDUH) 
which is a major source of information for the use of illicit drugs and mental 
health issues among the U.S. population aged 12 or older \cite{samhsa18}. 
The NSDUH is a comprehensive public survey that includes more than 2600 
variables on a diverse array of questions related to the use, misuse, and 
abuse of substances including alcohol, tobacco, prescription medications, and 
illicit drugs. In addition to typical demographic information, the survey
includes self-reported measures on items related to physical health, mental 
health (e.g., depression, anxiety, suicidal ideation), counseling, and drug 
and alcohol treatment. Data from the NSDUH has been used for identifying 
groups at high risk for substance use, and the co-occurrence of substance 
use and mental health disorders. The target outcome for the study was any 
misuse or abuse of prescription pain relievers and the predictor variables of 
interest were demographic variables, medication usage, and use of illicit drugs. 

%%%%%%%%%%%%%%%%%%%%%%%%%%%%%%%%%%%%%%%%%%%%%%%%%%%%%%%%%%%%%%%%%%%%%%%%%%%%%%%%

\subsection{Classification Models}

\subsubsection{Linear Models}

As the statistician George Box stated, "All models are wrong, but some models 
are useful" \cite{box05}. There are advantages and limitations for selecting
any model, but logistic regression is one of the most powerful and interpretable 
models for classification. Logistic regression models the conditional 
distribution of probabilities for a binary response (e.g., $Pr(Y=k | X=x)$) 
as a combination of a set of predictor variables \cite{james13, raschka17}. 
The decision boundary for the logistic regression classifier is a linear 
function of the input; a binary classifier separates two classes using a line, 
plane, or hyperplane \cite{muller17}. Given that the probability values for 
the outcome range between 0 and 1, predictions can be made based on a default 
value. For example, a default value of `Yes` could be predicted for any 
individual for whom the probability of pain reliever misuse and abuse is 
greater than fifty-percent, $Pr(PRLMISAB) > 0.5$. Logistic regression uses 
a maximum likelihood method to predict the coefficient estimates that 
correspond as closely as possible to the default state. In other words, the 
model will predict a number close to one for individuals who have misused or 
abused pain relievers and a number close to zero for individuals who have 
not. The distribution of conditional probabilities in the logit model has an 
S-shaped curve. The coefficients estimates obtained from a logistic regression 
model, ($beta_0$, $beta_1$ ... $beta_k$) are selected in order to maximize the 
likelihood function, and are interpreted as an indication of the log-odds 
change in the outcome variable that is associated with a one-unit increase 
in a predictor variable ($X_i$...$X_j$). 

%\begin{figure}[!ht]
%  \centering\includegraphics[width=\columnwidth]{images/Figure2.pdf}
%  \caption{Conditional Probabilities for Predicting Class Labels with 
%  a Logistic Regression Model \cite{raschka17}}
%  \label{f:Figure2}
%\end{figure}

\emph{Linear discriminant analysis} (LDA) is an alternative approach to 
estimating the probabilities which models the distribution of predictors 
separately in each of the response classes and then uses the Bayes' theorem 
to flip these into estimates for $Pr(Y=k | X=x)$ \cite{james13}. The term 
linear in LDA refers to the the discriminant functions being linear functions 
of the predictors. For distributions assumed to be normal (i.e., multivariate 
Guassian), LDA provides a model that is similar in form to logistic regression, 
but more stable. LDA is also preferred for outcomes with more than two response 
classes. The important assumptions for LDA are, first, a common covariance 
matrix for all classes, and second, the class boundaries are linear functions 
of the predictors. \emph{Quadratic Discriminant Analysis} (QDA) is an approach 
that assumes each class has its own covariance matrix and the decision 
boundaries are quadratically curvilinear in the predictor space \cite{kuhn13}. 
LDA is less flexible as a classifier than QDA, but can perform better with 
relatively few training observations or when the majority of predictors in the 
data represent discrete categories. QDA is recommended over LDA with a very 
large training set or when the decision boundary between two classes is 
non-linear. 

%%%%%%%%%%%%%%%%%%%%%%%%%%%%%%%%%%%%%%%%%%%%%%%%%%%%%%%%%%%%%%%%%%%%%%%%%%%%%%%%

\subsubsection{Non-linear Models}

The performance of linear classifiers suffers when there is a non-linear 
relationship between the predictors and target outcome. With training
observations that can be separated by hyperplane, the maximal marginal 
classifier provides the maximum distance (i.e., margin) from each observation
to the hyperplane \cite{james13}. The test observations are classified based 
on which side of the hyperplane they fall, but in many cases no separating 
hyperplane exists. The \emph{support vector classifier} (SVC) extends the 
maximal margin classifier by using a soft margin that allows a small number 
of observations to be misclassified on the wrong side of hyperplane 
\cite{kuhn13, cortes95}. The observations that fall directly on the margin or 
on the wrong side of the hyperplane are called `support vectors'. The parameter 
`C' indicates the number of observations that can violate the margin; if $C>0$, 
no more than C observations can be on the wrong side of the hyperplane. 
SVC addresses the problem of non-linear boundaries between classes by 
enlarging the feature space with higher order (e.g., quadratic, cubic, 
polynomial) functions of the predictors. 

\begin{figure}[!ht]
  \centering\includegraphics[width=\columnwidth]{images/Figure3.pdf}
  \caption{Decision Boundary and Support Vectors for Support Vector Machine 
  SVM Classifier with Nonlinear Kernel \cite{muller17}}
  \label{f:Figure3}
\end{figure}

\emph{Support vector machines} (SVM) are an extension of SVC that use a 
kernel trick to reduce computational load. The radial basis function (RBF) 
kernel (i.e., Guassian kernel) is one of the most commonly used approaches. 
In training the model, only a subset of data points is used to construct the 
decision boundary, namely the support vectors that lie on the border that 
separates the two classes. In predicting classes for new observations, the 
algorithm calculates the distance to each of the support vectors measured 
by the Guassian kernel \cite{muller17}. Figure 1 shows an example of the 
non-linear decision boundary obtained with SVM using the RBF kernel; the 
decision boundary is a smooth curve and the support vectors are the large 
points in bold outline. Even with the default settings, the RBF kernel 
provides a decision boundary that is decidedly non-linear. The parameters 
for SVM are `C', which regulates the importance of each data point, and 
`gamma' which controls the width of the Guassian kernel. A small value of 
C indicates a restricted model in which the influence of each data point is 
limited and the algorithm adjusts to the majority of data points. With larger 
values of C, more importance is given to each data point and the model tries 
to correctly classify as many training observations as possible, which results 
in more curvature in the decision boundary. Large values of gamma mean that 
only close values are relevant for classification, resulting in a smooth 
decision boundary. Small values of gamma mean that far points are similar. 
If the values of both C and gamma are large, each point can have an large 
influence in a small region, which produces a choppy decision boundary. 
If the values of C and gamma are both small, the decision boundary becomes 
close to linear.

%%%%%%%%%%%%%%%%%%%%%%%%%%%%%%%%%%%%%%%%%%%%%%%%%%%%%%%%%%%%%%%%%%%%%%%%%%%%%%%%

The Bayes' rule was mentioned above in relation to LDA; in this section, the 
\emph{naive Bayes classifier} is considered as a non-linear model.
The Bayes theorem (equation 1) is represented by a set of probabilities to 
represent the following question: Based on a given set of predictors, what
is the probability than an outcome belongs to a particular class?

\begin{equation}
  \ P(Y=cl|X) = \frac{P(Y)*P(X|Y=cl)}{P(X)}\
\end{equation}

The prior probability, P(Y), is the expected probability of a given class 
based on what is known (e.g., rate of disease in the population). P(X) 
is the probability of the predictor variables. The conditional probability,
$P(X=cl|Y)$, is the probability of observing the predictor variables for data 
associated with a given class. The naive Bayes model is based on the 
assumption that all the predictor variables are independent, although this 
is not always realistic. The conditional probabilities are calculated based 
on the probability densities for each individual predictor \cite{kuhn13}. 
For categorical predictors, the observed frequencies in the training set data 
can be used to determine the probability distributions. The prior probability 
allows us to tilt the final probability toward a particular class. Class 
probabilities are created and the predicted class is one associated with the 
largest class probability. Despite the somewhat unrealistic assumption of 
independence among predictors, the naive Bayes model is computationally quick, 
even with large training sets, and performs competitively compared to other 
models. The naive Bayes model encounters issues when dealing with frequencies 
or probabilities equal to zero, especially for small sample sizes. In addition, 
as the number of predictors increases relative to sample size, the posterior 
probabilities will become more extreme.

%%%%%%%%%%%%%%%%%%%%%%%%%%%%%%%%%%%%%%%%%%%%%%%%%%%%%%%%%%%%%%%%%%%%%%%%%%%%%%%%

\emph{Neural Networks} are powerful models for classification and 
regression based on theories about connectivity in the brain \cite{kuhn13}. 
The pstudy considers a simple method called multilayer perceptrons 
(MLP) as a feed-forward neural network \cite{muller17, raschka17}. 
The outcome is modeled by an intermediary set of unobserved variables called 
hidden units, which are linear combinations of the original predictors. 
Each hidden unit is a combination of some or all of the predictors which 
are then transformed by a nonlinear function (e.g., sigmoidal). A neural 
network usually has multiple hidden units used to model the outcome. 
The MLP classifier computes weights between the inputs and the hidden layers, 
and weights between the hidden layers and the output. After computing each 
hidden unit, the output is modeled by a nonlinear combination of the hidden 
units. The nonlinear function allows the neural network to fit more 
complicated functions than a linear model. Neural networks are sensitive to 
the scaling of the features and can require extensive data preprocessing. 
There are several ways to modify the complexity of a neural network: by 
selecting the number of hidden layers, the number of units within each 
layer, and the regularization parameter (L2) which shrinks the weights 
towards zero. The feature weights provide an estimate of feature importance.
Although neural networks can capture information in large amounts of data 
with very complex models, some limitations are that they tend to overfit 
data used to train the model, and can be difficult to interpret. Neural
networks may work best with homogenous datasets where the predictor variables
all have similar meanings \cite{muhuri13}. For datasets with many different
kinds of features, tree-based methods may provide a better approach.

%%%%%%%%%%%%%%%%%%%%%%%%%%%%%%%%%%%%%%%%%%%%%%%%%%%%%%%%%%%%%%%%%%%%%%%%%%%%%%%%

\subsubsection{Tree-based Models} Decision trees are based on a hierarchy of 
\emph{`if-else'} questions starting from a root node and proceeding through a 
series of binary decisions or choices. Each node in the tree represents either 
a question or a terminal node (i.e.,leaf) that contains the outcome. Applied to 
a binary classification task, the decision tree algorithm learns the sequence
of if-else questions that arrives at the outcome most quickly. For continuous 
features, questions are expressed in the form: ``Is feature x larger than 
value y?'' In constructing the tree, the algorithm searches through all 
possible tests and finds a solution that is most informative about the target 
outcome \cite{muller17}. The recursive branching process yields a binary tree 
of decisions, with each node representing a test for a single feature. This 
process of partitioning is repeated until each leaf in the decision tree 
contains only a single target. Prediction for a new data point proceeds by 
checking which region of the partition the point falls in, and predicting the 
majority in that feature space. The main advantage of tree models is that they 
require little adjustment and are easy to interpret. A drawback is that they 
can lead to complex models which are highly overfit to the training data. 
Prepruning`can help reduce overfitting by limiting the maximum depth of the 
tree, or the maximum number of leaves. Another approach is to set the minimum 
number of points in a node required for splitting. Decision trees work well 
with features measured on different scales, or with data that has a mix of 
binary and continuous features. 

%%%%%%%%%%%%%%%%%%%%%%%%%%%%%%%%%%%%%%%%%%%%%%%%%%%%%%%%%%%%%%%%%%%%%%%%%%%%%%%%

\emph{Random Forests} is an ensemble approach that combines many simple trees 
that each overfit the data in different ways. By building many trees and 
averaging their results, random forests helps to reduce overfitting. In 
constructing the forests, the user selects the number of trees to build 
(e.g., 1000). Randomness is introduced using a bootstrapping method that 
repeatedly draws random samples of size n from the data set (with replacement).  
The decision trees are build on these random samples of the same size, with 
some points missing and some data points repeated \cite{muller17,raschka17}.
The algorithm makes a random selection of p- features, and uses a different 
set of features at each node branch. These processes ensure that all of the 
decision trees in the random forest are different. Random forests is one of 
the most widely used supervised learning algorithms that works well without 
very much parameter tuning or scaling of data. A limitation is that Random 
forests do not perform well with high-dimensional data, or data that is 
sparse such as text data.

%%%%%%%%%%%%%%%%%%%%%%%%%%%%%%%%%%%%%%%%%%%%%%%%%%%%%%%%%%%%%%%%%%%%%%%%%%%%%%%%

\emph{Gradient Boosted trees} is another ensemble method that combines 
multiple decision trees in a serial fashion, where each tree tries to correct 
for mistakes of the previous one \cite{muller17}. Gradient boosted regression 
trees use strong pre-pruning, with shallow trees of a depth of one to five. 
Each tree only provides a good estimate of part of the data; combining many 
shallow trees (i.e., ``weak learners'') iteratively improves performance. 
In addition to pre-pruning and the number of trees, an important parameter 
for gradient boosting is the \emph{learning rate} which determines how strongly 
each tree tries to correct for mistakes of previous trees. A high learning rate
produces stronger corrections, allowing for more complex models. Adding more 
trees to the ensemble also increases model complexity. Gradient boosting
and random forests perform well on similar tasks and data. A common 
approach is to first try random forests and then include gradient boosting 
to improve model accuracy. 

%%%%%%%%%%%%%%%%%%%%%%%%%%%%%%%%%%%%%%%%%%%%%%%%%%%%%%%%%%%%%%%%%%%%%%%%%%%%%%%%

\subsection{Metrics for Evaluating Prediction Models}

Evaluating the performance of learning algorithms is helpful for selecting 
the best model for a problem given the available data. Binary classification 
is assessed in terms of the successful assignment of observations to one of 
two classes: positive or negative. Medical testing outcomes are often used to 
illustrate classification decisions and errors. A person can be either 
diagnosed with an illness or not, and the person can actually have the 
illness or not. In the present case, individuals were classified as having 
previously misused and abused pain relievers or not, and the positive class
represents self-reported pain reliever misuse and abuse ever (PRLMISEVR).
The model predictions will be either correct or incorrect in relation to
observed outcomes. Model performance is typically evaluated using accuracy 
which is the number of correct predictions divided by the total number of 
all samples. Any model cannot make perfect predictions as mistakes are 
always to be found. For example, a negative instance can be labeled 
as positive: a person who has never misused or abused pain relievers may be 
classified as having done so (i.e., 'false positive'). Conversely, a positive 
instance may be classified as negative: a person who has misused and abused 
pain relievers may be labeled as never having done so (i.e., 'false negative').

%%%%%%%%%%%%%%%%%%%%%%%%%%%%%%%%%%%%%%%%%%%%%%%%%%%%%%%%%%%%%%%%%%%%%%%%%%%%%%%%

\begin{table}
  \caption{Confusion Matrix for Evaluating Classification Model Performance}
  \label{tab:freq}
  \begin{tabular}{llll}
    \toprule
     &  &  Actual Outcome & \\
    \midrule
     Predicted & Outcome & No Misuse & PRL Misuse \\
    \midrule
     & No Misuse & \textbf{True Negative} & False Positive \\
    \midrule
     & PRL Misuse & False Negative & \textbf{True Positive} \\
    \bottomrule
  \end{tabular}
\end{table}

Classification errors and correct decisions can be represented in a 
\emph{confusion matrix} (Table 1) that indicates the correspondence between 
predicted and actual outcomes. The confusion matrix is a two-by-two array in 
which the columns correspond to the actual observed classes and the rows 
correspond to the predicted classes. The main diagonal indicate the number of 
correctly classified samples (\emph{true positive, true negative}, while the 
other entries represent the number of samples in one class that were mistakenly 
classified as another class. Classification models are evaluated using several 
measures including recall, precision, and the $f_1$-score \cite{wiki18}. 
\emph{Recall} or ``sensitivity'' measures how many positive samples are 
captured by the positive predictions ( \(\frac{TP}{TP+FN}\) ), and is used 
when we want to identify all positive samples while avoiding false negatives. 
\emph{Precision} or the ``positive predictive value'', measures how many of 
the samples predicted as positive are actually positive  
( \(\frac{TP}{TP+FP}\) ), and is used as a metric when the goal is to limit 
the number of false positives. The \emph{$f_1$-score} or `f-measure' 
(equation 2) provides the harmonic mean of precision and recall. The 
$f_1$-score can be a better metric than accuracy in datasets with 
imbalanced classes, where one class is much more frequent than the other 
class, as it takes recall and precision into account \cite{muller17}.

\begin{equation}
  \ f_1 score = 2*\frac{Precision*Recall}{Precision+Recall}\
\end{equation}

 \begin{figure}[!ht]
  \centering\includegraphics[width=\columnwidth]{images/Figure1.pdf}
  \caption{K-Nearest Neighbors Classifier Accuracy for Training Set and 
  Testing Set as a function of Number of Neighbors}
  \label{f:Figure1}
\end{figure}

%%%%%%%%%%%%%%%%%%%%%%%%%%%%%%%%%%%%%%%%%%%%%%%%%%%%%%%%%%%%%%%%%%%%%%%%%%%%%%%%

\subsubsection{Training and Test Set Accuracy}

In constructing and evaluating predictive models it is important to select 
a model that performs well not only with data used to train the model, but 
also with new observations. A common procedure is to partition the sample data
into a \emph{training set} and a \emph{testing set} of observations that is 
set aside and used to evaluate model performance. By convention, 
approximately 70 to 80 percent of observations are used in the training set 
and the remaining observations are held in the testing set. Two main problems 
can occur in evaluating model performance: overfitting and underfitting. 
In the case of \emph{overfitting}, a model can have high accuracy on the 
training set but perform poorly with new data in the test set because the 
model is `over-fit' to the training data, By contrast, ``underfitting'' 
occurs when a model performs poorly with the training data but has higher 
accuracy with the new observations in the test set. One of the simplest 
classification models, K-Nearest neighbors is used as an illustrative example. 
classifies observations by assigning the label that is most frequent among the 
`k' number of nearest training samples. K is a parameter selected by the user. 
The accuracy of the KNN classifier for the training set and testing set is 
plotted as a function of the parameter k-neighbors in Figure 1. The plot 
shows that increased accuracy on the testing set is associated with 
decreased training set accuracy, and conversely, increase accuracy on the
training set is related to decreased test set accuracy. The best model 
optimizes testing set accuracy and strikes a balance between the problems of 
overfitting and underfitting. In the case of KNN, performance on the test set 
increased only slightly between 2 and 4 neighbors, but did not improve much 
beyond 5 neighbors. Therefore, a KNN model with k=4 neighbors provides 
a reasonable solution for the data. 

%%%%%%%%%%%%%%%%%%%%%%%%%%%%%%%%%%%%%%%%%%%%%%%%%%%%%%%%%%%%%%%%%%%%%%%%%%%%%%%%

\subsubsection{Study Goals}

The goal of the study was to compare the performance of ten different 
classifier models to identify which model is best for classifying pain 
reliever misuse and abuse and to identify important features that 
predict the misuse and abuse of prescription opioids.

%%%%%%%%%%%%%%%%%%%%%%%%%%%%%%%%%%%%%%%%%%%%%%%%%%%%%%%%%%%%%%%%%%%%%%%%%%%%%%%%

\begin{table*}[ht]
  \caption{Summary of Variables in the NSDUH 2015-16 Aggregated Data Set}
  \label{tab:freq}
  \begin{tabular}{ll}
    \toprule
    \textit{Target Outcome} & Label \\
    \midrule
    Prescription Opioid Pain Reliever Misuse and Abuse (Likert scale: 0-12)& PRLMISAB  \\
    \midrule
    \textit{Predictor Variables}&   \\
    \midrule
    Age Category (1=12-17 years, 2=18-25, 3=26-34, 4=35-49, 5=50 and older)& AGECAT \\
    Biological Sex (0=Male, 1=Female)& SEX  \\
    Marital Status (0=Unmarried, 1=Divorced, 2=Widowed, 3=Married)& MARRIED  \\
    Education (1=H.S. or Less, 2=H.S. Grad., 3=Some College,  4=College Grad.)& EDUCAT  \\
    Size of City/Metropolitan Region (1=Rural, 2=Small, 3=Large)& CTYMETRO  \\
    Health Problems Aggregated  (Likert scale: 0-10)& HEALTH  \\
    Mental Health, Aggregated: adult depression, emotional distress (Likert scale: 0-10)& MENTHLTH  \\
    Treatment for Drugs and Alcohol in past year, Aggregated (Likert scale: 0-5)& TRTMENT  \\
    Mental Health Treatment, Aggregated (Likert scale: 1-10)& MHTRTMT  \\
    Tranquilizer use, past year, Aggregated (Likert scale: 0-5)& TRQLZRS \\
    Sedative use, past year, Aggregated (Likert scale: 0-5)& SEDATVS  \\
    Heroin use, past year, Aggregated (Likert scale: 0-5)& HEROINUSE  \\
    Cocaine and Crack Cocaine Use in past year, Aggregated  (Likert scale: 0-5)& COCAINE  \\
    Amphetamine and Methamphetamine Use in past year, Aggregated (Likert scale: 0-5)& AMPHETMN  \\
    \bottomrule
  \end{tabular}
\end{table*}

%%%%%%%%%%%%%%%%%%%%%%%%%%%%%%%%%%%%%%%%%%%%%%%%%%%%%%%%%%%%%%%%%%%%%%%%%%%%%%%%

\section{Method}

\subsection{The Data}

The NSDUH public data files for 2015 and 2016 were downloaded from the
Substance Abuse and Mental Health Data Archive (SAMHDA) \cite{samhsa18}. 
The data sets were extracted and saved as data frame objects in a python 
interactive notebook \cite{mckinney17, vanderplas17}. The 2015 NSDUH data set 
consisted of N=57146 respondents with 2667 variables and the 2016 data set had 
N=57897 individuals and 2665 variables, resulting in a total sample of N=114043 
observations (53873 male, 60165 female). As described in the NSHUD codebook, 
the sampling design was weighted across states by population size, drawing more 
heavily from eight states with the largest populations (CA, FL, IL, MI, NY, OH, 
PA, TX), for a representative distribution that accounts for approximately 
48 percent of the U.S. population. For 2015, the weighted survey screening 
response rate  was 81.94\% and the weighted interview response rate was 71.2\% 
\cite{samhsa18}. Identifying information in the public use files is collapsed 
(e.g., age categories); variables related to ethnicity, immigration status, and 
state identifiers are removed to ensure confidentiality. The data frames were 
subset by column to select approximately 90 variables that included common 
demographic characteristics, physical health, mental health, medication usage, 
and illicit drug use. Inconsistencies in the data were detected and removed by
the following steps: (a) Remove missing values (i.e., NaN); (b) Recode blanks, 
non-responses, or legitimate skips (e.g., 99, 991, 993) to zero; (c) Recode 
dichotomous responses (e.g., No=0, Yes=1); (d) Recode categorical variables 
to be consistent with amount or degree  (e.g., 1=`Low', 2=`Med', 3=`High'). 
(e) Outliers were identified and excluded from the data set (n=5).

%%%%%%%%%%%%%%%%%%%%%%%%%%%%%%%%%%%%%%%%%%%%%%%%%%%%%%%%%%%%%%%%%%%%%%%%%%%%%%%%

\subsubsection{Aggregated Variables}

Related features were combined to created aggregated variables. For example, 
responses for overall health (reverse scored), previous diagnosis of STDs, 
hepatitis, HIV, cancer, or any hospitalizationa were combined into a single 
`HEALTH` variable that indicated history of health problems. A mental health 
variable (MENTHLTH) aggregated responses for adult depression, emotional 
distress, suicidal thoughts or plans. Binary responses for ten of the most 
commonly used prescription pain medications (e.g., Hydrocodone, Oxycodone, 
Tramadol, Morphine, Fentanyl, Oxymorphone, Demerol, Hydromorphone) were
aggregated into a variable for any prescription opioid pain reliever use 
(ANYPRLUSE) in the past year. The majority of questions related to substance 
use had dichotomous responses that were summed to create single measures for: 
Tranquilizers, Sedatives, Heroin, Cocaine, and Amphetamines. Because 
hallucinogens varied greatly in type and potency (e.g., marijuana, psilocybin, 
MDMA, LSD), they were not included in the analysis. Variables for drug 
treatment and mental health treatment combined responses for any inpatient 
care, outpatient care, treatment at a clinic, emergency room visits, or 
hospital stays. The target variable was a dichotomous measure of any previous 
pain reliever misuse or abuse. The subset data frame consisted of 20 features 
and 114038 observations and was exported to CSV file. Three variables (PRLANY, 
PRLMISAB, HEROINEVR) were highly correlated with other variables and excluded 
from the analysis . Table 2 shows the list of 15 predictor variables used for 
constructing the classification models. 


%%%%%%%%%%%%%%%%%%%%%%%%%%%%%%%%%%%%%%%%%%%%%%%%%%%%%%%%%%%%%%%%%%%%%%%%%%%%%%%%

\subsubsection{Classifier Models}

The dataset was divided into the training set and test set using a $75\%/25\%$  
split ($n_train$=85538, $n_test$=28510). The same general procedure was used in 
constructing each classification model: (i) The model was fit to the training 
set; (ii) New values were predicted on the holdout scores in the testing set; 
and (iii) Model performance was evaluated in a confusion matrix. Performance 
metrics|accuracy, sensitivity (i.e., recall), precision| were obtained from 
the confusion matrix; the $f_1-score$ was derived from the values for recall 
and precision (equation 1). The logistic regression classifier, LDA, QDA, 
decision trees classifier, random forests classifier, gradient boosted 
classifier were constructed using the caret package. The KNN classifier, 
support vector classifier (SVC), naive Bayes classifier, and neural net 
(multilayer perceptron) were constructed using scikit-learn. The main 
parameter settings for each model are reported in Table 4. 

%%%%%%%%%%%%%%%%%%%%%%%%%%%%%%%%%%%%%%%%%%%%%%%%%%%%%%%%%%%%%%%%%%%%%%%%%%%%%%%%

\section{Results}

\subsection{Exploratory Data Analysis}

Of the total sample of N=114038 respondents from 2015-2016, 27\% (n=30790) 
reported taking any pain relievers in the past year (13405 males, 17383 
females). Approximately 11\% (n=12305) of the sample disclosed that they 
had previously misused or abused prescription pain relievers at some point 
(6239 males, 6066 females). The percent of individuals reporting pain reliever 
misuse and abuse in the training set and test sets was the same as in the 
full sample. Although, more females reported using any prescription opioid 
pain relievers in the past year than males, roughly equal proportions of 
males and females reported misusing or abusing opioid pain relievers. 
Figure 4 shows that the proportion of pain reliever misuse and abuse was 
higher among younger individuals, between the ages of 12 to 25, than 
individuals over 26 years, and decreased among individuals age 36 and older. 
Only 2\% of individuals (n=2266) disclosed ever using heroin (1320 males; 
946 females). As shown in Figure 5, the proportion of pain reliever misuse 
and abuse was higher among individuals who reported using heroin than those 
who had not, which is consistent with previous findings that indicate a 
relationship between the misuse and abuse of prescription opioids and 
heroin use \cite{muhuri13, unick13}.

\begin{figure}[!ht]
  \centering\includegraphics[width=\columnwidth]{images/Figure4.pdf}
  \caption{Proportion of Individuals Reporting Pain Reliever Misuse and Abuse
  as a function of Heroin Use Ever}
  \label{f:Figure4}
\end{figure}

\begin{figure}[!ht]
  \centering\includegraphics[width=\columnwidth]{images/Figure5.pdf}
  \caption{Proportion of Individuals Reporting Pain Reliever Misuse and Abuse
  and Heroin Use Ever as a function of Sex}
  \label{f:Figure5}
\end{figure}
 
%%%%%%%%%%%%%%%%%%%%%%%%%%%%%%%%%%%%%%%%%%%%%%%%%%%%%%%%%%%%%%%%%%%%%%%%%%%%%%%%
 
\begin{table*}[ht]
  \caption{Confusion Matrices and Performance Metrics for Predictive Models of 
  Pain Reliever Misuse and Abuse}
  \label{tab:freq}
  \begin{tabular}{llllllll}
    \toprule
    Model& & Confusion Matrix & & Accuracy & Sensitivity & Precision & F1-Score \\
    \midrule
    K-Nearest Neighbors & & No Misuse & PRL Misuse &  &  &  & \\
     & No Misuse & 25114 & 320 & 89.5\% & 0.900 & 0.870 & 0.870 \\
     & PRL Misuse & 2609 & 467 &  &  &  & \\
    \midrule
    Logistic Regression & & No Misuse & PRL Misuse &  &  &  & \\
     & No Misuse & 25002 & 2509 & 90\% & 0.986 & 0.909 & 0.946 \\
     & PRL Misuse & 344 & 654 &  &  &  & \\
    \midrule
    Linear Discriminant Analysis (LDA) & & No Misuse & PRL Misuse &  &  &  & \\
     & No Misuse & 24668 & 2239 & 89.8\% & 0.973 & 0.917 & 0.944 \\
     & PRL Misuse & 678 & 924 &  &  &  & \\
    \midrule
    Quadratic Discriminant Analysis (QDA) & & No Misuse & PRL Misuse &  &  &  & \\
     & No Misuse & 23165 & 1780 & 86.1\% & 0.914 & 0.9929 & 0.921 \\
     & PRL Misuse & 2181 & 1383 &  &  &  & \\
    \midrule
    Support Vector Classifier (SVM) & & No Misuse & PRL Misuse &  &  &  & \\
     & No Misuse & 25201 & 233 & 90.4\% & 0.900 & 0.890 & 0.880 \\
     & PRL Misuse & 2514 & 562 &  &  &  & \\
    \midrule
    Naive Bayes & & No Misuse & PRL Misuse &  &  &  & \\
     & No Misuse & 25345 & 3133 & 89\% & 0.999 & 0.890 & 0.941 \\
     & PRL Misuse & 1 & 30 &  &  &  & \\
    \midrule
    Neural Network (MLP) & & No Misuse & PRL Misuse &  &  &  & \\
     & No Misuse & 25088 & 346 & 90.6\% & 0.910 & 0.890 & 0.880 \\
     & PRL Misuse & 2366 & 710 &  &  &  & \\
    \midrule
    Decision Trees & & No Misuse & PRL Misuse &  &  &  & \\
     & No Misuse & 25042 & 2572 & 89.9\% & 0.988 & 0.907 & 0.946 \\
     & PRL Misuse & 304 & 591 &  &  &  & \\
    \midrule
    Random Forests & & No Misuse & PRL Misuse &  &  &  & \\
     & No Misuse & 25026 & 2298 & 90.1\% & 0.987 & 0.909 & 0.946 \\
     & PRL Misuse & 320 & 665 &  &  &  & \\
    \midrule
    Gradient Boosted Trees & & No Misuse & PRL Misuse &  &  &  & \\
     & No Misuse & 25398 & 36 & 89.9\% & 0.900 & 0.890 & 0.895 \\
     & PRL Misuse & 2856 & 220 &  &  &  & \\
    \bottomrule
  \end{tabular}
\end{table*}

%%%%%%%%%%%%%%%%%%%%%%%%%%%%%%%%%%%%%%%%%%%%%%%%%%%%%%%%%%%%%%%%%%%%%%%%%%%%%%%%

\begin{table}
  \caption{Classification Models for Predict Pain Reliever Misuse 
  and Abuse and Main Parameters}
  \label{tab:freq}
  \begin{tabular}{ll}
    \toprule
    Model & Main Parameter \\
    \midrule
    K-Nearest Neighbors & Number of Neighbors = 4 \\
    Logistic Regression & NA \\
    Naive Bayes & Cost C = 0.01 \\
    Neural Network & Hidden Layers = 2 \\
    Decision Trees & Tree-Depth = 4 \\ 
    Random Forests & Number of Trees = 1000 \\
    Boosted Trees & Learning Rate = 0.01 \\ 
    \bottomrule
  \end{tabular}
\end{table}

%%%%%%%%%%%%%%%%%%%%%%%%%%%%%%%%%%%%%%%%%%%%%%%%%%%%%%%%%%%%%%%%%%%%%%%%%%%%%%%%

%%%%%%%%%%%%%%%%%%%%%%%%%%%%%%%%%%%%%%%%%%%%%%%%%%%%%%%%%%%%%%%%%%%%%%%%%%%%%%%%

\subsection{Comparison of Classifier Models}

Performance of the classification models was evaluated in the confusion 
matrices reported in Table 3, which includes model accuracy, sensitivity 
(i.e., Recall), precision, and the $f_1$-score. In terms of accuracy, neural
networks (MLP) and support vector classifier had the best performance; 
however, as described in the introduction, the $f_1$-score is a better 
performance metric than accuracy with datasets that have unbalanced classes 
because it takes into account both sensitivity and precision. There is a 
clear indication of unbalanced classes in the NSDUH data given that the 
proportion of respondents who had never misused or abused pain relievers 
was much greater than the proportion who had. Approximately 11\% of 
respondents in both the training and test sets reported misusing and abusing 
opioid pain relievers. Using the $f_1$-score as the preferred metric, logistic
regression, decision trees, and random forests performed better than the other 
classification models, and were ited with the same $f_1$-score. 

The main parameter settings for the classification models are presented in 
Table 4


\subsubsection{Logistic Regression versus Tree models}
 
the parameter estimates for the logistic regression classifier model are 
presented in Table 5, sorted in descending order by z-score. The z-score
provides an indication of the relative influence of each predictor on the 
dependent variable. The model identified cocaine, amphetamines, tranquilizers,
mental health, age category, and heroin use as six of the most influential.

All of the features were significant predictors of pain reliever misuse and abuse
except for mental health which was non-significant.   The model was rerun leaving out 
the non-significant variables, mental health 
treatment, which did not change the sign or significant of the parameter estimates. 
Age category, sex, and size of city of metropolitan area were negatively related
to the outcome variables, whereas the remaining significant predictors were 
positively related to pain reliever misuse and abuse

to predict pain reliever misuse and abuse
from the set of predictor variables. The logistic regression classifier was fit to 
the training data using the caret package, and the model was validated on the test 
data. Together, all of the variables entered into the model significantly predicted 
pain reliever misuse and abuse; only mental health treatment was a non-significant
predictor. 


%%%%%%%%%%%%%%%%%%%%%%%%%%%%%%%%%%%%%%%%%%%%%%%%%%%%%%%%%%%%%%%%%%%%%%%%%%%%%%%%

\begin{table}
  \caption{Coefficient Estimates for Logistic Regression Model 
  of Pain Reliever Misuse and Abuse fit to the Training Set}
  \label{tab:freq}
  \begin{tabular}{llll}
    \toprule
    Predictor&  Estimate& Std. Error& z-value  \\    
    \midrule
    (Intercept)& -2.960 &   0.048 & -61.10 ***  \\
    COCAINE  &    0.690 &   0.019 &  36.91 ***  \\
    AMPHETMN &    0.608 &   0.021 &  29.92 ***  \\
    TRQLZRS  &    0.381 &   0.014 &  27.29 ***  \\
    MENTHLTH &    0.136 &   0.006 &  22.42 ***  \\
    AGECAT   &   -0.216 &   0.013 & -16.75 ***  \\
    HEROINUSE&    0.798 &   0.052 &  15.47 ***  \\  
    EMPLOY18 &    0.222 &   0.016 &  13.72 ***  \\
    TRTMENT  &    0.195 &   0.019 &  10.47 ***  \\
    SEDATVS  &    0.281 &   0.029 &   9.63 ***  \\   
    HEALTH   &    0.103 &   0.012 &   8.27 ***  \\
    EDUCAT   &    0.104 &   0.013 &   8.22 ***  \\   
    SEX      &   -0.186 &   0.026 &  -7.24 ***  \\
    CTYMETRO &   -0.068 &   0.013 &  -5.03 ***  \\
    MARRIED  &    0.051 &   0.012 &   4.10 ***  \\
    MHTRTMT  &   -0.024 &   0.018 &  -1.32     \\
    \bottomrule
    Note: *** p-value $<$ 0.001. &  &  
  \end{tabular}
\end{table}

%%%%%%%%%%%%%%%%%%%%%%%%%%%%%%%%%%%%%%%%%%%%%%%%%%%%%%%%%%%%%%%%%%%%%%%%%%%%%%%%

\subsubsection{Decision Tree Classifier}

The Decision Tree Classifier package in Scikit-Learn was used to build the 
tree model, pre-pruning was applied with a maximum depth of 4, which means 
the algorithm split on four consecutive questions. The training set accuracy 
of the pruned tree was 0.902 and test set accuracy was 0.902. Figure 9 shows 
a partial view of the decision tree classifier of prescription opioid misuse

The Decision Tree Classifier model was build using the default setting of a 
fully developed tree until all leaves are pure. Accuracy on the
training set was 0.99 and test set accuracy was 0.974.
Without restricting 
their depth, decision trees can become complex; unpruned trees are prone to 
overfitting and do not generalize well to new data. 
pre-pruning was applied, with 
a maximum depth of 4, which means the algorithm split on four consecutive
questions. Training set accuracy of the pruned tree was 0.985 and test set
accuracy was 0.984. Even with a depth of 4, the tree can become a bit complex.
Limiting the depth of 
tree decreases overfitting, which results in lower training set accuracy, 
but improved performance on the test set. 

One way to interpret a decision tree it by following the sample numbers 
represented at the test split for each node. The classifier algorithm 
selected Cocaine Use as the root node of the decision tree. 
The branch to the left  side of the tree represents samples with a score equal 
to or less than 1.5 (n=40956), whereas the branch to the right represents 
samples with a Cocaine Use score greater than 1.5 (n=1903). The second split 
on the right occurs for X, with n=1443 having a score less than or equal 
to 3.5, and n=460 respondents with a PRL score greater than 3.5. In other 
words, of those respondents who reported relatively high Cocaine use, a small
portion also reported relatively high Prescription Opioid PRL use.


Feature importance is a common summary function that rates how important 
each feature is for the classification decisions made in the algorithm. 

Each feature is assigned an importance value between 0 and 1; with a value of 
1 indicating the feature perfectly predicts the target and a value of 0 meaning 
that the feature was not used at all. Feature importance values also always 
sum to 1. A feature may have a low importance value because another feature 
encodes the same information. 

As Figure 9 shows, the decision tree classifier selected Cocaine Use as the 
root note. At the second node, on the branch to the right n=5015 samples were further divided according to 

heroin use, with n=1913 having a score greater 
than 0.5 (any Heroin Use). At the third node on the right branch, samples were selected according to Tranquilizer medication use, with n=1419 scoring positively. On the left branch, the second node selected was Drug Treatment, with n=2844 respondents scoring
positively that they had received Drug Treatment. Feature importance of
the decision tree classifier selected Cocaine Use as the most informative
feature for Prescription Opioid PRL Misuse. Following afterwards, 
Tranquilizer Use, Drug Treatment, and Heroin Use were tied for second place. 

\begin{figure}[!ht]
  \centering\includegraphics[width=\columnwidth]{images/Figure6.pdf}
  \caption{Decision Tree Classification of Pain Reliever Misuse and Abuse}
  \label{f:Figure6}
\end{figure}


%%%%%%%%%%%%%%%%%%%%%%%%%%%%%%%%%%%%%%%%%%%%%%%%%%%%%%%%%%%%%%%%%%%%%%%%%%%%%%%%
\subsubsection{Random Forests Classifier}

Random forests reduces overfitting by building many trees and averaging 
their results. The important parameters are the number of trees (100), 
the number of data points for bootstrap sampling, and the maximum number of 
features considered at each node which determines how random each tree is. 
The algorithm can look at all of the features in the dataset or only a 
limited number. A high value for maximum-features will produce trees that 
are very similar and will fit the data easily based on the most distinctive 
features, whereas a low value will produce trees that are very different 
from each other, which also reduces over-fitting. 

with smaller values of max-features resulting in trees in the random forest 
that are very different from each other. This analysis applied a random forest 
consisting of 100 trees 

The test set accuracy was 0.984. 

Often the default settings for random forests work well, but we can apply
pre-pruning as with a single tree, or adjust the maximum number of features. 
Feature importance for random forests is computed by aggregating the feature 
importance over trees in the random forest, and random forests gives
non-zero importance to more features than a single tree. 

The Random Forest Classifier package in Scikit-Learn was used to classify
Prescription Opioid PRL Misuse as the target variable, with 100 trees. 

compensating for some of their shortcomings of overfitting.

To reduce overfitting, pre-pruning could be implemented by reducing the 
maximum depth, or by reducing 
the learning rate. 

Although random forests models are more accurate than a single tree and can
compensate for some of the shortcomings of overfitting, a limitation of this 
approach is that it can be difficult to interpret an ensemble of trees apart 
from the measure of feature importance provided in the output. Single trees 
are still useful for visually representing the decision process.

Feature importance was a primary interest for identifying features related to
prescription opioid abuse. Figure 11 shows the feature importance for the 
gradient boosting classifier tree. As Figure 11 shows, several features were 
important for classifying prescription opioid misuse, and contrary to the 
random forests, gradient boosting selected Tranquilizer use as the most 
informative feature. Following closely in importance were Heroin Use and Age 
Category. Tied for fourth place were Cocaine Use and Treatment, with Mental 
Health (depression) coming in fourth in terms of feature importance. This 
result illustrates that several features are important for understanding 
Prescription Opioid Misuse, and the relations among features may be complex.

%%%%%%%%%%%%%%%%%%%%%%%%%%%%%%%%%%%%%%%%%%%%%%%%%%%%%%%%%%%%%%%%%%%%%%%%%%%%%%%%

\begin{table}
  \caption{Feature Importance for Random Forest Model: Mean Decrease in Gini
  Score}
  \label{tab:freq}
  \begin{tabular}{ll}
    \toprule
    Predictor&  Mean Decrease in Gini Score  \\    
    \midrule
    COCAINE   & 1284.84 \\
    AMPHETMN  &  830.93 \\
    MENTHLTH  &  748.94 \\ 
    TRQLZRS   &  638.85 \\
    AGECAT    &  524,89 \\
    HEROINUSE &  514.82 \\
    HEALTH    &  501.38 \\
    EDUCAT    &  450.63 \\
    CTYMETRO  &  329.09 \\
    MHTRTMT   &  327.66 \\
    MARRIED   &  293.53 \\
    EMPLOY18  &  285.78 \\ 
    TRTMENT   &  236.56 \\
    SEX       &  197.71 \\
    SEDATVS   &  139.23 \\
    \bottomrule
  \end{tabular}
\end{table}

The Gradient Boosting Classifier used the default setting of 100 trees, of 
maximum depth of 3, and a learning rate of 0.1. 

Accuracy for the test set was 0.893. Gradient boosting typically improves test 
set accuracy by using many simple models iteratively. The model accuracy for 
gradient boosting was no better than random forests, and this is because the 
default parameter settings were used. Further parameter tuning is needed 
to improve model performance.

%%%%%%%%%%%%%%%%%%%%%%%%%%%%%%%%%%%%%%%%%%%%%%%%%%%%%%%%%%%%%%%%%%%%%%%%%%%%%%%%

\subsubsection{Accuracy and Interpretability}


Simple models had lower accuracy but provide more interpretable solutions 
than complex models which can yield higher accuracy but are often harder 
to interpret. 

One of the simplest classifier algorithms, K-Nearest Neighbors (KNN)
classifies a new data point based on the Euclidean distance to its 
nearest neighbors and provides a solution that is easy to understand. 
However, KNN does not perform well with a large number of features 
(100 or more) or sparse datasets.

SVMs often perform well, but are sensitive to parameter settings 
and scaling of the data. 

\begin{figure}[!ht]
  \centering\includegraphics[width=\columnwidth]{images/Figure7.pdf}
  \caption{Neural Net Classifier: Multilayer Perceptron with Single Hidden Layer}
  \label{f:Figur7}
\end{figure}



%%%%%%%%%%%%%%%%%%%%%%%%%%%%%%%%%%%%%%%%%%%%%%%%%%%%%%%%%%%%%%%%%%%%%%%%%%%%%%%%

\section{DISCUSSION}

% Summarize main findings

The results show that rates of prescription opioid use, misuse, and abuse are
much higher than use of illicit opioids such as heroin and fentanyl. The use 
of Hydrocodone (Vicodan) was double the rate of Oxycodone use (Oxycodone) 
across almost all age groups. The use of traditional prescription opioids 
was greater than reported use of synthetic opioids. Illicit drug use was 
highest for respondents between the ages of 18 to 25. In terms of mental 
health, more individuals between 18 to 25 years reported experiencing a major 
depressive episode (in adulthood) than any other age group. 

The large majority of respondents (89\$ percent) had not misused 
prescription opioid 
pain relievers or used heroin. 

Of those individuals who reported 
misusing prescription opioid pain relievers, almost twice as many had also
used heroin than had not (see Figure 1), which partially supports the 
hypothesis that prescription opioid use is associated will use of illicit 
opioids such as heroin. 
 
 Recent statistics from the CDC show that heroin use
has increased among most demographics groups, with an average estimated rate 
of approximately 2.6 percent between 2011-2013 \cite{cdc16}.

The rate of heroin use reported in the NSDUH-2015 sample was 1.6 percent. 
Therefore, it seems that the actual rate of heroin use in the U.S. population 
may not be accurately reflected in this sample. 

 
%%%%%%%%%%%%%%%%%%%%%%%%%%%%%%%%%%%%%%%%%%%%%%%%%%%%%%%%%%%%%%%%%%%%%%%%%%%%%%%%

\subsection{Comparison of Classifier Models}

Several classifier algorithms were used to identify relevant features for 
predicting heroin use and prescription opioid misuse. Comparing the performance 
of different algorithms is helpful for  selecting the best model. 

Test set
accuracy was comparable across models for both Heroin Use (0.98) and 
Prescription Opioid PRL Misuse (0.89-0.90). Logistic Regression provided the
feature coefficients for different values of the regularization parameter C. 
The Decision Tree classifier provided a easy to use, interpretable visual of
the decisions involved at each step of classification. Random forests provides
a more reliable indication of features importance than a single tree, 
whereas the gradient boosting classifier included additional tuning 
parameter for a more powerful model and more interpretable analysis of
feature importance. 

Each classifier method provides a different level of
analysis. 
logistic regression showed that Treatment had the highest coefficient value, but 
the tree based models each differed in selecting the most important features. 

Decision trees indicated that Cocaine Use was most informative, the random 
forests classifier selected health as the most important feature, and the 
gradient boosting model selected 
Tranquilizer use as most informative of prescription opioid PRL misuse. 

The different models each have their advantages and limitations, logistic regression provides the coefficients, but random forests and gradient boosting are helpful for 
identified sets of important features.

%%%%%%%%%%%%%%%%%%%%%%%%%%%%%%%%%%%%%%%%%%%%%%%%%%%%%%%%%%%%%%%%%%%%%%%%%%%%%%%%
\subsection{Combine Limitations and Extensions}

Surveys data may be biased to some degree, but measures of confidentiality and 
anonymity help to assure more accurate disclosures. 

The main goal of this project was to 


identify features relevant for predicting 
opioid addiction by classifying cases according to heroin use. 

A limitation of survey data is that responses may be biased by under-reporting or minimizing the use of illicit or illegal substances. People may also be reluctant to disclose mental health issues or health problems (e.g., STDs, HIV status, suicide attempts). 

It is possible that this sample is representative of the frequency of opioid use 
and misuse in the larger population.


A limitation is that the project dataset was constructed as a subset of 
features from the NSDUH-2015 data. 
Ninety attributes were selected out of 2666 features in the original dataset, 
and many features were combined to create aggregated variables for health, 
mental health, prescription opioid misuse and abuse, drug treatment, mental health
treatment. 

Future research could include a more comprehensive selection of
features to identify the set of features relevant for predicting opioid
dependency and addiction. 


%%%%%%%%%%%%%%%%%%%%%%%%%%%%%%%%%%%%%%%%%%%%%%%%%%%%%%%%%%%%%%%%%%%%%%%%%%%%%%%%
\subsection{Extension to Big Data}

The methods used in this project could be extended to better approximate big 
data for predicting opioid use in the following ways: (1) Include a larger 
selection of features from the attributes in the NSDUH-2015 dataset; (2) 
Include survey data from previous years (e.g., 2005-2015) for a larger sample;  
and (3) Obtain a broader sample from the population of patients who are 
taking prescribed opioid medications. The most immediate step would be to 
include additional features for use with the classifier models. Additional 
data from the NSDUH was downloaded from previous years (2012 to 2014); 
preliminary examination of the data revealed inconsistencies in questions 
and prescription opioid medications that would need to be resolved in order 
to combine data from multiple years. Data cleaning can be a time consuming 
process, but important for obtaining usable data. Unfortunately, owing to 
constraints of time for completing the project, it was not possible to
integrate data from previous years into the project dataset. In working with
big data, there are there are several steps involved in the consolidation of 
data from multiple sources into a single dataset (in addition to data 
cleaning), which include extraction, integration, and aggregation of features  
\cite{rahm00}. A future study could integrate data from different years, 
using a broader set of features, with more inclusive sample representative
of the larger population, and integrate data from multiple sources. 

%%%%%%%%%%%%%%%%%%%%%%%%%%%%%%%%%%%%%%%%%%%%%%%%%%%%%%%%%%%%%%%%%%%%%%%%%%%%%%%%
\subsection{Opioid Addiction Epidemic}

If the crisis of opioid addiction were an epidemic like other illnesses caused 
by biological contagion, its spreading or diffusion could be measured by the 
proportion of infected in the population, those yet to be infected, and the 
rate of transmission. Although drug addiction has many similar characteristics 
to other chronic medical illnesses, but there are unique challenges to the 
treatment of addiction \cite{marsch12, swendson16}. According to the classical 
conditioning theory of addiction, situational cues or events can elicit a 
motivational state underlying the relapse to drug use. Following treatment, 
many addicted individuals return to the same environments associated with 
their drug use. Addictive behavior can be reinstated by exposure to drug-
related cues or stressors in the environment \cite{shaham03}, putting
individuals in recovery at risk for relapse and possible overdose \cite{johnson11}. 
Furthermore, the structure of social contact networks can influence epidemic 
spreading or diffusion \cite{pastor01, watts98}, as person-to-person contact 
and indirect associations may facilitate the spread of drug use and addictive 
behavior. In the case of prescription opioids, anyone is potentially at 
risk of misusing opioids and becoming addicted. Rather than labeling people 
as addicted or not addicted, it may be useful to consider people as more or 
less susceptible to misusing or abusing opioid pain medication, leading to
a greater or lesser susceptability to becoming dependent or addicted to 
opioids. 

%%%%%%%%%%%%%%%%%%%%%%%%%%%%%%%%%%%%%%%%%%%%%%%%%%%%%%%%%%%%%%%%%%%%%%%%%%%%%%%%
\section{Conclusion}

The present study compared the results of several classification models of 
pain reliever misuse and abuse. A general conclusion is that 

The results do not provide sufficient evidence to rule out alternative hypotheses. 

Several features were identified as important for classifying pain reliever
misuse and abuse, including Cocaine Use, Amphetamine Use, tranquilizer use,
age category, overall health,

Prescription opioid misuse and abuse was associated with heroin use: of 
the individuals who reported misusing prescription opioid medications, twice 
as many reported having used heroin than those who said that they had not used 
heroin. the importance of heroin use was a predictor of pain reliever misuse and
abuse was limited, given the very small number of individuals who reported 
using heroin.

additional evidence is needed to address these question. 

Predictive modeling is a useful approach for classifying individuals as 
having misused or abuse opioid pain relievers and may help inform 
efforts to address the opioid crisis and reduce risk of overdose deaths. 


%%%%%%%%%%%%%%%%%%%%%%%%%%%%%%%%%%%%%%%%%%%%%%%%%%%%%%%%%%%%%%%%%%%%%%%%%%%%%%%%
\begin{acks}

Portions of this paper were completed as part of a course project in Big Data 
Applications and Analytics taugt by Dr. Gregor von Laszewskin at Indiana 
University in the Fall 2017. Thanks to the Professor von Laszewski and the
Teaching Assistants, Juliette Zurick, Miao Jiang, Hungri Lee, Grace Li, and 
Saber Sheybani Moghadam for helpful comments and feedback.

\end{acks}

\bibliographystyle{unsrt} %%ACM-Reference-Format%%
\bibliography{report} 


%%%%%%%%%%%%%%%%%%%%%%%%%%%%%%%%%%%%%%%%%%%%%%%%%%%%%%%%%%%%%%%%%%%%%%%%%%%%%%%%
\appendix

\section{Code References}
All code, notebooks, files, and folders for this project can be found in the

%%%%%%%%%%%%%%%%%%%%%%%%%%%%%%%%%%%%%%%%%%%%%%%%%%%%%%%%%%%%%%%%%%%%%%%%%%%%%%%%


%\input{issues}

\end{document} 
